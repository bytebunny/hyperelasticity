\section{Task C}
\label{sec:task-c}

The equivalent internal forces are given as
\begin{equation} \tag{9.15b}
  \label{eq:int-force}
  \ubar{\bm{T}}_{a}^{(e)} = \int_{v^{(e)}} \utilde{\sigma} \cdot \ubar{\nabla} N_{a}\, dv
\end{equation}
For a CST element of thickness \(t\) the same expression in local coordinates reads 
\begin{equation}
  \label{eq:int-force-local}
  \ubar{\bm{T}}_{a}^{(e)} = \int \int \utilde{\sigma} \cdot \ubar{\nabla} N_{a} \, t 
  \, dx \, dy =
  \int_{-1}^{1} \int_{-1}^{1} \utilde{\sigma} \cdot \ubar{\nabla} N_{a} \, t 
  \det \left( \frac{\partial \ubar{\bm{x}}}{\partial \ubar{\bm{\xi}}} \right) \,
  d\xi \, d\eta
\end{equation}
Apply Gaussian quadrature rule to compute the integral numerically:
\begin{equation}
  \label{eq:int-force-numer}
  \ubar{\bm{T}}_{a}^{(e)} \approx
  \sum_{i=1}^{\text{nip}} W_{i} \utilde{\sigma} \cdot \ubar{\nabla} N_{a} \, t 
  \det \left( \frac{\partial \ubar{\bm{x}}}{\partial \ubar{\bm{\xi}}} \right),
\end{equation}
where \(W_{i}\) is a weight for each of the integration points.
For a CST element, the number of integration points is 1 with coordinates
\(\xi = 1/3\), \(\eta = 1/3\) and the weight \(W = 0.5\).

The Matlab implementation of this task can be found in 
\texttt{get\_intern\_eq\_forces.m} (see section \ref{app:matlab-code}).

%%% Local Variables:
%%% mode: latex
%%% TeX-master: "../main"
%%% End:
